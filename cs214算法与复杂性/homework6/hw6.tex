\documentclass[12pt,a4paper]{article}
\usepackage{ctex}
\usepackage{amsmath,amscd,amsbsy,amssymb,latexsym,url,bm,amsthm}
\usepackage{epsfig,graphicx,subfigure}
\usepackage{enumitem,balance}
\usepackage{wrapfig}
\usepackage{mathrsfs,euscript}
\usepackage[usenames]{xcolor}
\usepackage{hyperref}
\usepackage[vlined,ruled,commentsnumbered,linesnumbered]{algorithm2e}

\newtheorem{theorem}{Theorem}
\newtheorem{lemma}[theorem]{Lemma}
\newtheorem{proposition}[theorem]{Proposition}
\newtheorem{corollary}[theorem]{Corollary}
\newtheorem{exercise}{Exercise}
\newtheorem*{solution}{Solution}
\newtheorem{definition}{Definition}
\theoremstyle{definition}


%\numberwithin{equation}{section}
%\numberwithin{figure}{section}

\renewcommand{\thefootnote}{\fnsymbol{footnote}}

\newcommand{\postscript}[2]
{\setlength{\epsfxsize}{#2\hsize}
	\centerline{\epsfbox{#1}}}

\renewcommand{\baselinestretch}{1.0}

\setlength{\oddsidemargin}{-0.365in}
\setlength{\evensidemargin}{-0.365in}
\setlength{\topmargin}{-0.3in}
\setlength{\headheight}{0in}
\setlength{\headsep}{0in}
\setlength{\textheight}{10.1in}
\setlength{\textwidth}{7in}
\makeatletter \renewenvironment{proof}[1][Proof] {\par\pushQED{\qed}\normalfont\topsep6\p@\@plus6\p@\relax\trivlist\item[\hskip\labelsep\bfseries#1\@addpunct{.}]\ignorespaces}{\popQED\endtrivlist\@endpefalse} \makeatother
\makeatletter
\renewenvironment{solution}[1][Solution] {\par\pushQED{\qed}\normalfont\topsep6\p@\@plus6\p@\relax\trivlist\item[\hskip\labelsep\bfseries#1\@addpunct{.}]\ignorespaces}{\popQED\endtrivlist\@endpefalse} \makeatother

\usepackage{listings}

\begin{document}
\noindent

%========================================================================
\noindent\framebox[\linewidth]{\shortstack[c]{
\Large{\textbf{Homework06}}\vspace{1mm}\\
CS214-Algorithm and Complexity, Spring 2018}}
\begin{center}

\footnotesize{\color{black} Name: 杨培灏  \quad Student ID: 516021910233}
\end{center}

\begin{enumerate}

\item  什么时候单源最短路径算法对于某些边权值为负的图是有效的?

\begin{solution}
	有以下两种情况:
	\begin{enumerate}
		\item 不能出现负权回路,如果出现负权回路,那么可以无限经过该回路将路径缩短,不会有最短回路。
		\item 单元最短路径算法是一个贪心算法,它成立的基础是当前最短路径是整个图中的最短路径。所以若存在负权,需要保证之前不是最短的路径加上负权的边后仍然大于最短路径。即不能改变已经确定的最短路径。
	\end{enumerate}
\end{solution}

\item 证明任意连通图中必然存在一个点,删除该点不影响图的连通性。用线性时间找到这个点

\begin{solution}
		证明:如果图中不存在回路,那么一定存在度为1的点,显然删除它不会影响连通性。如果不存在度为1的点,那么说明原图为点数大于$2$的双连通图,那么在双连通图中,只有关节点删除后会影响连通性。
	\begin{center}
	\begin{algorithm}[H]
		\caption{deleteVertice()}
		\KwIn{点数n,点集V,邻接表edge}
		\KwOut{删除点}
		initialize a degree array of size $n$\\
		\For{$i=1\to n$}{
			\For{$e_{ij} in edge$}{
				degree[i]+=1;
			}
			\If{degree[i]}{
				output i;\\
				return;
			}
		}
		Biconnected\_Components();\\
		//求解节点HIGH值和DFS顺序下的节点标号DFS\_Number\\
		\For{$i=1\to n$}{
			\If{v[i].HIGH!=v.DFS\_Number}{
				output i;//不是关节点就可以删除\\
			}
		}
	\end{algorithm}
	\end{center}
	\begin{center}
		\begin{algorithm}[H]
			\caption{Biconnected\_Components()}
			\KwIn{V,E(无向连通图点集、边集),节点数n,DFS树根节点v}
			\For{v in V}{
				v.DFS\_Number=0;\\
			}
			DFS\_N=n;
			BC(v);
		\end{algorithm}
	\end{center}
	\begin{center}
		\begin{algorithm}[H]
			\caption{BC()}
			\KwIn{DFS开始节点v}
			v.DFS\_Number=DFS\_N;//对新点标号\\
			DFS\_N=DFS\_N-1;\\
			Stack.push(v);\\
			v.HIGH=v.DFS\_Number;//初始HIGH都是自身值
			\For{e(v,w) in edge}{
				Stack.push(e);//由于是无向图边将进栈两次
				\If{w is not the parent of v}{
					\If{w.DFS\_Number=0}{
						BC(w);//DFS\_Number为0还没有被遍历到
						\If{$w.HIGH\leq v.DFS\_Number$}{
							\While{Stack.top()!=v}{
								Stack.pop();		
							}
							//移除栈中的边和点,他们组成的子图是一个双连通图;\\
							Stack.pop();\\
							Stack.push(v);\\
							//因为v还可能同时是另一段分支的一部分\\
							v.HIGH=max(v.HIGH,w.HIGH);\\
						}
					}
				}
				\Else{
					//此时(v,w)将该点连到了parent\\
					v.HIGH=max(v.HIGH,w.DFS\_Number);\\
				}
			}
		\end{algorithm}
	\end{center}
	
	\textbf{复杂度分析:}算法中度数的统计遍历了边集,所以时间为$\mathcal{O}(|E|)$,双连通图的判定时间为$\mathcal{O}(|E|+|V|)$,总的时间复杂度为$\mathcal{O}(|E|+|V|)$。
\end{solution}

\item 设G是有向非循环图,其所有路径最多含$k$条边,设计线性时间算法,将所有顶点分为$k+1$组,每一组中任意两个点之间不存在路径

\begin{solution}
	先由某个初始节点DFS遍历一次,标记是否访问过。第一次读到叶子节点就进行一次分配,回溯后进入刚刚经过的分支的兄弟分支,由于仍然在同一棵树上,之前的连通性仍然保证,只需要看之后的部分,需要注意如果读到已经访问过的节点在分组时需要避开。
	
	如果仍有点未分配,说明图从初始节点遍历是不连通的,那么还会有未访问的点和边。找到这些独立的路径,同样DFS操作。注意此时求出的路径上的点尽管确实与最初的节点不连通,但是可能与之前去掉的某些点连通,这时可以利用之前记录的各点是否访问过,在分配组时避开被访问过的点。重复算法直到所有点都被分配。
	
	由于我们在遍历过程中按在路径上的次序分配点,所以不可能有放不到$k+1$个组中的点。下方算法需要更换点执行,直到所有点都被访问过。
	\begin{center}
		\begin{algorithm}[H]
			\caption{DFSdivide(v)}
			\KwIn{边集E,点集V,DFS开始节点v}
			initialize a group array of size $(k+1)$;\\
			initialize a flag array of size $(k+1)$;\\
			//判断新路径经过了哪些访问过的点;\\
			\For{v in V}{
				v.depth=0;\\
				v.visited=False;\\
			}
			depth=0;\\
			\If{!v.visited}{
				depth=depth+1;\\
				v.depth=depth;\\
				v.visited=True;\\
				group[v.depth].push(v);\\
			}
			\Else{
				flag[v.depth];\\
				depth=depth+1;\\
				从之前的group[v.depth]中删除掉这条路径上的点;\\
				//这里比较难实现还没有想出来\\
			}
			\For{child in v.children}{
				divide(child);
			}
		\end{algorithm}
	\end{center}
	
	\textbf{复杂度分析:}遍历一次图需要遍历所有点和边,所以总的时间复杂度为$\mathcal{O}(|E|+|V|)$
	
	如果对时间限制不严格,那么另一种思路是我们可以使用传递闭包算法,根据邻接矩阵,去构造一个传递闭包的矩阵,这个矩阵直接反应了两点间的连通性。这样的时间复杂度是$\mathcal{O}(|V|^3)$。那么可以根据这个矩阵再进行点的分配,这个过程需要遍历传递闭包算法构造出的矩阵,最终总的复杂度会达到$\mathcal{O}(|V|^3/w)$,在并行处理的规模w非常大,图非常密集的情况下,可以强行解释它近似于$\mathcal{O}(|E|)$。传递闭包的算法实现起来较简单如下:
	\begin{center}
		\begin{algorithm}[H]
			\caption{transiveClosure()}
			\KwIn{邻接矩阵E,点数n}
			\For{$i=1\to n$}{
				\For{$j=1\to n$}{
					\If{E[i][j]==True}{
						\For{$k=1\to n$}{
							\If{E[j][k]==True}{E[i][k]=True;}
						}
					}	
				}
			}
		\end{algorithm}
	\end{center}
\end{solution}

\item 给定连通无向图G,以及3条边$a$,$b$,$c$,在线性时间内判断G中是否存在一个包含$a$和$b$但不含$c$的闭链。
\begin{solution}
	由于一条边仅会出现在一个双连通分支中,而两边同属于一个双连通分支当仅当存在闭链包含两条边,所以考虑将判断闭链转化为判断双连通分支,直接移除点$c$及其相连的边,此时如果$a$,$b$不属于一个双连通分支,故确定原命题不成立。如果$a$,$b$属于一个双连通分支,那么该双连通分支中必存在包含$a$,$b$的闭链。
	\begin{center}
		\begin{algorithm}[H]
			\caption{ABwithoutC()}
			\KwIn{点集V,边集E}
			remove C from V and related edge from E;\\
			ABwithoutC();\\
			判断是否存在双连通分支;\\
		\end{algorithm}
	\end{center}
	\begin{center}
		\begin{algorithm}[H]
			\caption{Biconnected\_Components()}
			\KwIn{V,E(无向连通图点集、边集),节点数n,DFS树根节点v}
			\For{v in V}{
				v.DFS\_Number=0;\\
			}
			DFS\_N=n;
			BC(v);
		\end{algorithm}
	\end{center}
	\begin{center}
		\begin{algorithm}[H]
			\caption{BC()}
			\KwIn{DFS开始节点v}
			v.DFS\_Number=DFS\_N;//对新点标号\\
			DFS\_N=DFS\_N-1;\\
			Stack.push(v);\\
			v.HIGH=v.DFS\_Number;//初始HIGH都是自身值
			\For{e(v,w) in edge}{
				Stack.push(e);//由于是无向图边将进栈两次
				\If{w is not the parent of v}{
					\If{w.DFS\_Number=0}{
						BC(w);//DFS\_Number为0还没有被遍历到
						\If{$w.HIGH\leq v.DFS\_Number$}{
							\While{Stack.top()!=v}{
								Stack.pop();		
							}
							//移除栈中的边和点,他们组成的子图是一个双连通图;\\
							Stack.pop();\\
							Stack.push(v);\\
							//因为v还可能同时是另一段分支的一部分\\
							v.HIGH=max(v.HIGH,w.HIGH);\\
						}
					}
				}
				\Else{
					//此时(v,w)将该点连到了parent\\
					v,HIGH=max(v.HIGH,w.DFS\_Number);\\
				}
			}
		\end{algorithm}
	\end{center}
	
	\textbf{复杂度分析:}双连通图的判定时间为$\mathcal{O}(|E|+|V|)$,所以算法的时间复杂度是线性的。
\end{solution}

\item 设计线性时间算法求树的最大匹配
\begin{solution}
	动态规划的方法。设n为点数,矩阵$dp$的大小为$n\times 2$,$dp[i][0]$为第$i$号节点不与子节点匹配时子树的最大匹配数,$dp[i][1]$为第$i$号节点与子节点匹配时的子树的最大匹配数,$dp$状态转移方程为$$dp[i][0]=\sum_{j\in children}max(dp[j][1],dp[j][0])$$$$dp[i][1]=\sum_{j\in children}max(dp[j][1],dp[j][0])-min_{j\in children}(max(dp[j][1],dp[j][0]))+dp[j][0]+1$$最后输出$max(dp[0][0],dp[0][1])$。
	结合状态转移方程和广度优先搜索易得到具体算法。
	
	\textbf{复杂度分析:}树形动态规划,遍历了每个节点,由于儿子数量不确定,时间复杂度难以确定。假如平均的儿子数量为c,节点数为|V|,那么时间复杂度为$\mathcal{O}(c|V|)$。
\end{solution}

\item 设$n$个选手进行单循环赛,比赛结果没有平局,且存储在一个矩阵中,设计$\mathcal(O)(n\log n)$的算法为这些选手排序,使得$P_1$胜$P_2$,$P_2$胜$P_3$,以此类推。
\begin{solution}
	在问题中,定义有向边$a\to b$表示选手$a$战胜选手$b$,那么题干可以表述为,由$n$个点构成的有序完全图,找到其中的哈密顿道路。
	首先说明哈密顿道路的存在性,引入两个定理:
	\begin{enumerate}
		\item 在$n(n\geq2)$阶有向图D中,如果所有有向边均用无向边代替,所得无向图中含生成子图$K_n$,则有向图中存在哈密顿道路。
		\item 设$G$是$n(n\geq3)$阶无向简单图,如果$G$中任何一对不相邻的顶点度数之和都大于等于$n$,则$G$是哈密顿图。
	\end{enumerate}
	由以上定理可知,$n(n\geq3)$阶有序的完全图中存在哈密顿道路。题干要求的$\mathcal(O)(n\log n)$应该是针对的边而言。因为如果是针对点的话,需要对每个点都只进行$\mathcal(O)(\log n)$的操作,而图是一个完全图,所以甚至连点连接的边的方向信息都获取不全,无法构造路径。
	下面是一个构造哈密顿道路的算法,由于我们知道这个道路必然存在,所以可以枚举每个点进行尝试。这里利用归纳法的思路,当我们得到了前n个点的哈密顿道路后,想要加入第n+1个点,会有三种情况,只存在它到每个点的道路,那么加入到头部即可;如果是只有其他点到它的道路,那么加到最后;如果均有,那么它一定可以找到一个空隙将它插入。其中第二种第三种均为向后延伸路径,可以合并。事实上这种数学归纳法的算法也证明了该图中必然存在哈密顿路径。
	\begin{center}
	\begin{algorithm}[H]
		\caption{Hamilton()}
		\KwIn{邻接矩阵E,点数n}
		\KwOut{路径path}
		initialize matrix next -1;\\
		\For{$i=0\to n$}{
			head=i;//以每个点为起点尝试查找\\  
			\For{$j=1\to n$}{  
				\If{i==j}{continue;}  
				\If{E[j][head]==1}{  
					//第一种:向前延伸路径 \\
					next[j]=head;\\  
					head=j; \\ 
				} 
				\Else{
					//第二种:插入延伸路径 \\  
					pre=head,pos=next[head];\\
					\While{$pos!=-1$ and $E[pos][j]==1$}{  
						pre=pos;  \\
						pos=next[pre]; \\
					}
					next[pre]=j; \\
					next[j]=pos; \\
				}  
			}
			//填写路径\\
			k=1;j=head; \\
			\While{$j!=-1$}{path[k++]=j;\\j=next[j]}
			\If{$path[n]!=-1$}{return path;}
			//路径最后一项也被填了\\
		}
	\end{algorithm}

	\textbf{复杂度分析:}算法的时间复杂性最坏情况下为$\mathcal{O}(|V|^3)$,由于完全图中边数为$|V|\times(|V|-1)/2$,所以可以视为$\mathcal{O}(|E|^{1.5})$。不是一个$\mathcal{O}(|E|\log |E|)$的算法。
	\end{center}
\end{solution}

\item 无向图$G$的顶点覆盖是指顶点集合$U$,$G$中每条边都至少有一个顶点在此集合中。设计线性时间算法为树寻找一个顶点覆盖,并且使该点集的规模尽量小。
\begin{solution}
	动态规划的方法。设n为点数,矩阵$dp$的大小为$n\times 2$,$dp[i][0]$为第$i$号节点不被覆盖时子树的最小点覆盖数,$dp[i][1]$为第$i$号节点被覆盖时子树的最小点覆盖数,$dp$状态转移方程为$$dp[i][0]=\sum_{j\in children}max(dp[j][0],dp[j][1])$$$$dp[i][1]=\sum_{j\in children}(dp[j][0])+1$$
	最后输出$min(dp[0][0],dp[0][1])$。
	结合状态转移方程和广度优先搜索易得到具体算法。
	
	\textbf{复杂度分析:}树形动态规划,遍历了每个节点,由于儿子数量不确定,时间复杂度难以确定。假如平均的儿子数量为c,节点数为|V|,那么时间复杂度为$\mathcal{O}(c|V|)$。
\end{solution}
\end{enumerate}

%========================================================================
\end{document}
