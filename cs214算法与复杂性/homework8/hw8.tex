\documentclass[12pt,a4paper]{article}
\usepackage{ctex}
\usepackage{amsmath,amscd,amsbsy,amssymb,latexsym,url,bm,amsthm}
\usepackage{epsfig,graphicx,subfigure}
\usepackage{enumitem,balance}
\usepackage{wrapfig}
\usepackage{mathrsfs,euscript}
\usepackage[usenames]{xcolor}
\usepackage{hyperref}
\usepackage[vlined,ruled,commentsnumbered,linesnumbered]{algorithm2e}

\newtheorem{theorem}{Theorem}
\newtheorem{lemma}[theorem]{Lemma}
\newtheorem{proposition}[theorem]{Proposition}
\newtheorem{corollary}[theorem]{Corollary}
\newtheorem{exercise}{Exercise}
\newtheorem*{solution}{Solution}
\newtheorem{definition}{Definition}
\theoremstyle{definition}


%\numberwithin{equation}{section}
%\numberwithin{figure}{section}

\renewcommand{\thefootnote}{\fnsymbol{footnote}}

\newcommand{\postscript}[2]
{\setlength{\epsfxsize}{#2\hsize}
	\centerline{\epsfbox{#1}}}

\renewcommand{\baselinestretch}{1.0}

\setlength{\oddsidemargin}{-0.365in}
\setlength{\evensidemargin}{-0.365in}
\setlength{\topmargin}{-0.3in}
\setlength{\headheight}{0in}
\setlength{\headsep}{0in}
\setlength{\textheight}{10.1in}
\setlength{\textwidth}{7in}
\makeatletter \renewenvironment{proof}[1][Proof] {\par\pushQED{\qed}\normalfont\topsep6\p@\@plus6\p@\relax\trivlist\item[\hskip\labelsep\bfseries#1\@addpunct{.}]\ignorespaces}{\popQED\endtrivlist\@endpefalse} \makeatother
\makeatletter
\renewenvironment{solution}[1][Solution] {\par\pushQED{\qed}\normalfont\topsep6\p@\@plus6\p@\relax\trivlist\item[\hskip\labelsep\bfseries#1\@addpunct{.}]\ignorespaces}{\popQED\endtrivlist\@endpefalse} \makeatother

\usepackage{listings}

\begin{document}
\noindent

%========================================================================
\noindent\framebox[\linewidth]{\shortstack[c]{
\Large{\textbf{Homework07}}\vspace{1mm}\\
CS214-Algorithm and Complexity, Spring 2018}}
\begin{center}

\footnotesize{\color{black} Name: 杨培灏  \quad Student ID: 516021910233}
\end{center}

\begin{enumerate}

\item {
	对凸多边形
	\begin{enumerate}
		\item 有多少种三角划分的方法?
		\begin{solution}
			对于凸多边形,假设我们已知任意$k$边形的三角划分数为$S_k$,其中$k\ge4$,特殊地,定义$S_2=1$和$S_3=1$。那么对于$k+1$边形,由于其轮廓上任意一条边都属于一个三角形,我们可以取一对相邻点记作$v_1$和$v_{k+1}$,沿边界标记其他点为$v_2,\ldots,v_k$,这对相邻点可以与$v_2,\ldots,v_k$中任意一点连接,将$k+1$边形分划为一个三角形和两个多边形的组合。那么有递推关系$S_{k+1}=S_2\times S_k+S_3\times S_{k-1}+\ldots+S_k\times S_2$。卡特兰数的解为$h_{n+1}={2n\choose n}/(n+1)$,而观察$S_{n}=h_{n-2}$。综上可以得到$S_{n}={2n-4\choose n-2}/(n-1)$。		
		\end{solution}
		\item 如何使对角线长度之和最小?
		\begin{solution}
			采用贪心法的思路,先计算每条对角线长度,并为它们排序。总是选择当前最短的一条对多边形进行划分,如果对角线出现交点,就按照顺序重新选择下一条加入三角划分的图中,直到选择了$n-3$条边。
			证明该贪心算法的正确性:在选取当前最短对角线$e_1$后,下一个被选中的是$e_2$,如果仍然存在其他对角线满足$e_1+e_2+\ldots+e'_{n-3}>e'_1+e'_2+\ldots+e'_{n-3}$,而注意到$e_1$是当前的最小值,则必有$e_i+e_j>e'_i+e'_j$,即$n-1$边形内部的一个三角形和与其拥有一条公共边的、顶点在多边形外部的三角形相比,周长较小,这是不可能的,所以不存在使得对角线长度更小的方法。
		\end{solution}
	\end{enumerate}
} 
	
\item 求n个矩形的交集,这些矩形的边都平行于坐标轴
\begin{solution}
	若采用分治法。这里指的是将矩形的集合一分为二,然后对两个子集中的矩形分别求交集,重复如此操作直到分割到基础情况。基础情况为两个多边形或一个多边形,如果是一个多边形,则直接返回该多边形;如果是两个多边形,返回他们的交集(可能为空集)。考虑两个凸多边形的交集仍然为凸多边形,仍然使用扫描线的思路,由凸集的性质,扫描线与一个凸多边形至多有两个交点,所以扫描线状态表最多需要保存4个点。设扫描线从左到右移动,求两个多边形的交集可以在$\mathcal{O}(n)$时间内完成。所以该算法的时间复杂度为$\mathcal{O}(n\log n)$。
	
	特别地,因为这些矩形的边都是平行于坐标轴的,因此我们可以给出一个类似求线段交点的线性时间算法。扫描线竖直与矩形边重合,从左向右移动。事件进度表为矩形的竖直边,扫描线扫描时顺序读取读到的边,如果所有矩形的左侧边未读完就读到了某个矩形右侧边,那么交集为空,否则读完左侧边后,只保留最后一条左侧边,这是可能存在的重合区域的左侧边,读到的第一条右侧边位可能存在的重合区域的右侧边。同理,在扫描时还可以,找到重合区域的上边界和下边界。扫描是线性的,所以算法的时间复杂度为$\mathcal{O}(n)$。但是如果实现扫描必须先给矩形的各边排序,则时间复杂度为$\mathcal{O}(n\log n)$,同时在排序时可以求出所有矩形各边的极值,找出重合部分,不必进行上面说的扫描。
	\begin{algorithm}
		\caption{findIntersection()}
		\KwIn{矩形集合R}
		//对所有矩形的各边排序并顺便求出重合部分,结束不必再扫描\\
		\If{最靠右的矩形左侧边x坐标>最靠左的矩形右侧边x坐标}{
			return $\emptyset$\\
		}
		\ElseIf{最靠上的矩形下侧边y坐标>最靠下的矩形上侧边y坐标}{
			return $\emptyset$\\
		}
		\Else{
			//扫描矩形并求出下面各个数值\\
			maxLeft,maxDown,minUp,minRight即为矩形交集\\
		}
	\end{algorithm}
\end{solution}
\end{enumerate}

%========================================================================
\end{document}
