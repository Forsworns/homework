\documentclass[12pt,a4paper]{article}
\usepackage{ctex}
\usepackage{amsmath,amscd,amsbsy,amssymb,latexsym,url,bm,amsthm}
\usepackage{epsfig,graphicx,subfigure}
\usepackage{enumitem,balance}
\usepackage{wrapfig}
\usepackage{mathrsfs,euscript}
\usepackage[usenames]{xcolor}
\usepackage{hyperref}
\usepackage[vlined,ruled,commentsnumbered,linesnumbered]{algorithm2e}
\renewcommand{\algorithmcfname}{算法}
\newtheorem{theorem}{Theorem}
\newtheorem{lemma}[theorem]{Lemma}
\newtheorem{proposition}[theorem]{Proposition}
\newtheorem{corollary}[theorem]{Corollary}
\newtheorem{exercise}{Exercise}
\newtheorem*{solution}{Solution}
\newtheorem{definition}{Definition}
\theoremstyle{definition}


%\numberwithin{equation}{section}
%\numberwithin{figure}{section}

\renewcommand{\thefootnote}{\fnsymbol{footnote}}

\newcommand{\postscript}[2]
{\setlength{\epsfxsize}{#2\hsize}
	\centerline{\epsfbox{#1}}}

\renewcommand{\baselinestretch}{1.0}

\setlength{\oddsidemargin}{-0.365in}
\setlength{\evensidemargin}{-0.365in}
\setlength{\topmargin}{-0.3in}
\setlength{\headheight}{0in}
\setlength{\headsep}{0in}
\setlength{\textheight}{10.1in}
\setlength{\textwidth}{7in}
\makeatletter \renewenvironment{proof}[1][Proof] {\par\pushQED{\qed}\normalfont\topsep6\p@\@plus6\p@\relax\trivlist\item[\hskip\labelsep\bfseries#1\@addpunct{.}]\ignorespaces}{\popQED\endtrivlist\@endpefalse} \makeatother
\makeatletter
\renewenvironment{solution}[1][Solution] {\par\pushQED{\qed}\normalfont\topsep6\p@\@plus6\p@\relax\trivlist\item[\hskip\labelsep\bfseries#1\@addpunct{.}]\ignorespaces}{\popQED\endtrivlist\@endpefalse} \makeatother

\usepackage{listings}
\lstset{
	tabsize=2 % sets default tabsize to 2 spaces
}
\begin{document}
\noindent

%========================================================================
\noindent\framebox[\linewidth]{\shortstack[c]{
\Large{\textbf{Homework03}}\vspace{1mm}\\
CS214-Algorithm and Complexity, Spring 2018}}
\begin{center}

\footnotesize{\color{black} Name: 杨培灏  \quad Student ID: 516021910233}
\end{center}

\begin{enumerate}

\item 设计算法将1到n2按顺时针方向由内而外填入一个$n \times n$的矩阵
\begin{gather*}
	\begin{bmatrix}
	25 & 10 & 11 & 12 & 13\\
	24 & 9 & 2 & 3 & 14\\
	23 & 8 & 1 & 4 & 15\\
	22 & 7 & 6 & 5 & 16\\
	21 & 20 & 19 & 18 & 17\\
	\end{bmatrix}\quad
\end{gather*}
\begin{solution}
	注意到$n$为奇数时,想填满一圈可以从左上角开始逆时针依次递减;而$n$为偶数时,一圈总是从右下角开始逆时针依次递减。因此可以写出以下伪代码:
	\begin{lstlisting}
		void solve(int n){
			initialize the martix m with the input numeral n;
			if(n % 2) fill_odd(n,0);
			else fill_even(n,n-1);
		}
		void fill_odd(int n, int k){
		//n是递归输入的规模,k是初始的填入位置(k,k)
			if(n == 1) {
				a[k][k] = 1;
			} else {
				for(int i=0; i<n; ++i) a[k+i][k]=n*n-i;
				for(int i=0; i<n; ++i) a[k+n-1][k+i]=n*n-n-i+1;
				for(int i=0; i<n; ++i) a[k+n-1-i][k+n-1]=n*n-2n-i+2;
				for(int i=0; i<n; ++i) a[k][k+n-1-i]=n*n-3n-i+3;
				fill(n-2,k+1);
			}
		}
		void fill_even(int n, int k){
			if(n == 2){
				a[k][k] = 4;
				a[k-1][k] = 3;
				a[k-1][k-1] = 2;
				a[k][k-1] = 1;
			} else {
				类似地填满一圈
				fill(n-2,k-1);
			}
		}
	\end{lstlisting}
\end{solution}


\item 给定集合S和实数x,判断x是否能够表示成S中两个不同元素之和,给出算法及时间复杂性(越快越好)。

\begin{solution}
如果集合中元素已经经过排序,首先通过二分法在集合$S$中寻找两个分界点$x/2$、$x$,或其附近的最小闭区间的端点,这里时间复杂度为$O(nlogn)$。由于要在集合中寻找两数使其和为$x$,则集合中大于$x$的部分可以完全舍去,不再讨论。

将$S$中小于$x$的部分由$x/2$分界点拆分为两个单调递增的有序序列,$A=(S_1,S_2,\ldots,S_{k1})$,$B=(S_{K1+1},\ldots,S_{k2})$,这里的$k2 \leq n$,$S_{k1} \leq x/2$,$S_{k1+1} \geq x/2$,$S_{k2} \leq x$。之后运行如下算法:

\begin{algorithm}[H]
	\SetAlgoLined
	\caption{$find\_elements(A,B)$}
	\KwIn{序列$A$,$B$}
	\KwOut{布尔值,判断x是否能够表示成S中两个不同元素之和}
	\eIf{$sizeof(A)<sizeof(B)$}{
		\While{$i<sizeof(A)$}{
			$result \leftarrow binary\_search(B,x-A_i)$;		
		}
	}{
		\While{$i<sizeof(A)$}{
			$result \leftarrow binary\_search(A,x-B_i)$;		
		}
	\textbf{return} $result$;
	}
\end{algorithm}

\textbf{复杂度分析:}在$A$、$B$中长度较小的一个进行遍历,在另一个当中使用二分法对应寻找遍历到的元素与x的差。遍历的过程为$\mathcal{O}(n)$,二分法查找为$\mathcal{O}(\log n)$,故该部分的复杂度为$\mathcal{O}(n\log n)$。算法总的时间复杂度为$\mathcal{O}(n\log n)$。

若尚未经过排序,则须增加一步排序,使用快速排序或归并排序的时间复杂度为$O(nlogn)$,总的时间复杂度不变。
\end{solution}


\item 给定整数数组$A[1,\leq ,n]$,相邻两个元素的值最多相差1。设$A[1]=x$,$A[n]=y$,并且$x<y$,输入$z$,$x \leq z \leq y$,判断$z$在数组$A$中出现的位置。给出算法及时间复杂性(不得穷举)

\begin{solution}
看错了,当成是递增序列且要找到所有位置:

因为$A[1]=x$,$A[n]=y$且$\forall i \in {1,\ldots,n} A[i+1]-A[n]\leq 1$,所以$\forall z$,$x\leq z\leq y$,$z\in A$。但是可能存在多个$z$在数组$A$中。定义一个加法$a\leq b$,$b\leq c$,$(a,b)+(b,c)=(a,c)$,$(null,null)+(b,c)=(b,c)$。

\begin{algorithm}[H]
	\SetAlgoLined
	\caption{$binary_search(start,end)$}
	\KwIn{搜索的开始下标start,搜索的结束下标end,全局变量数组A,搜索目标z}
	\KwOut{(start,end)}
	\eIf{$start<end$}{
		$mid=(start+end)/2$;
		
		\If{$z<A[mid]$}{
			\textbf{return} $binary\_search(start,mid)$;
		}
		\ElseIf{$z>A[mid]$}{
			\textbf{return} $binary\_search(mid,end)$
		}
		\Else{
			\textbf{return} $binary\_search(start,mid)+binary\_search(mid+1,end)$;
		}	
	}{
		\eIf{$z=A[start]$}{
			\textbf{return} $(start, end)$;
		}{
			\textbf{return} $(null, null)$;
		}
	}
\end{algorithm}

即如果数组某区间正中间的值与$z$相等,则左右两侧均需要搜索。

\textbf{复杂度分析:}定义的求和操作在常数时间内可以完成。总体上,算法在最差的情况下,即如果$A$中有$n-1$个x,搜索$x$,则需要搜索$n-1$次。复杂度为$\mathcal{O}(n)$。最好情况下,如果每次均可以均分区间,那么算法和一般的二分搜索相同,算法复杂度为$\mathcal{O}(\log n)$。
\end{solution}

\begin{solution}
微信群里说明是任意序列,并且只需要找到一个,这种情况下类似上面所述,$z$是一定可以被找到的,并且事实上第一个元素完全可以由二分法查找找到。
	
证明:由于相邻数字最多相差1,所以可以认为在整数域上序列是连续的。则任意一个区间,如果$left<z<right$,那么在区间$(left,right)$间,至少存在一个等于$z$的值。于是可以得出二分法查找依然有效。

	\begin{algorithm}[H]
		\SetAlgoLined
		\caption{$binary\_search(start,end)$}
		\KwIn{搜索的开始下标start,搜索的结束下标end,全局变量数组A,搜索目标z}
		\KwOut{(start,end)}
		\eIf{$start<end$}{
			$mid=(start+end)/2$;\\
			\eIf{$z<A[mid]$}{
			\textbf{return} $binary\_search(start,mid)$;	
			}{
			\textbf{return} $binary\_search(mid+1,end)$;
			}
		}{\textbf{return} $start$;}
		
	\end{algorithm}
	
	\textbf{复杂度分析:}二分法查找的复杂度为$\mathcal{O}(\log n)$。
\end{solution}

\item 用分治法找到数组中的最大数和最小数,若数组规模为2的幂,证明需要的比较次数为$3n/2-2$。

\begin{solution}
	每次将数组分划成两部分,直到只剩两个元素可以直接比较得到最大最小值。
	
	\begin{algorithm}[H]
		\SetAlgoLined
		\caption{$min\_max(start,end)$}
		\KwIn{数组A,开始下标start,结束下标end}
		\KwOut{(min,max)}
		\If{$start==end$}{
			\textbf{return} $(A[start],A[start])$;
		}
		\ElseIf{$start==end+1$}{
			\eIf{$A[start]<A[end]$}{
				\textbf{return} $(A[start],A[end])$; 
			}{
				\textbf{return} $(A[end],A[start])$;
			}
		}\Else{
			$mid=(start+end)/2$;\\
			$(min1,max1)=min\_max(start,mid)$;\\
			$(min2,max2)=min\_max(mid,end)$;\\
			\eIf{$min1<min2$}{$min=min1$;}{$min=min2$};
			\eIf{$max1<max2$}{$max=max1$;}{$max=max2$};
			\textbf{return} $(min,max)$;
		}
	\end{algorithm}

	\textbf{复杂度分析:}若数组规模为2的幂,则每次对半划分数组,恰好可以得到两个规模为2的幂的数组,最后求得上述算法比较的次数为$2n-2$。
	
	如果采用尾递归的方式,若数组规模为2的幂,首先对两个元素的形式通过一次比较可以得到最大值最小值。若已知$k-2$个元素的最大值和最小值,那么对$k$个元素,可以先比较后两个元素,是一次比较,然后两数中的最大值最小值再分别和前$k-2$个元素的最大值最小值比较,又是两次,于是递推公式为$f(k)=f(k-2)+3$。于是数组规模为2的幂次时,比较次数为$f(2^k)=f(2)+3\times(2^{k-1}-1) (n=2^k)$。所以比较次数为$f(n)=3n/2-2$。
\end{solution}

\item 给定长度为$n$的字符串$S$,求$S$中最长回文字符串的长度,例如abcdefedchijip中最长的回文字符串为cdefedc,长度为$7$

\begin{solution}
	动态规划的思想,$dp[i][j]$表示$i$到$j(i\leq j)$的子串为回文字符串,则状态转移方程为$$dp[i][j]=\left\{
	\begin{array}{ll}
	true&{i=j}\\
	S[i]=S[j]&{i=j-1}\\
	dp[i][j]\text{ and }S[i+1]=S[j-1]&{i<j-1}
	\end{array}
	\right\}$$
	\begin{algorithm}[H]
		\SetAlgoLined
		\caption{$palindromic()$}
		\KwIn{输入字符串$S$}
		\KwOut{回文字符串长度$ans$}
		\For{$i=n \to 1$}{
			\For{$j=n \to i$}{
				\If{$i=j$}{
					$dp[i][j]=1$;		
				}
				\ElseIf{$i=j-1$}{
					$dp[i][j]=(S[i]==S[j])$;
				}
				\Else{
					$dp[i][j]=dp[i][j]\text{ and }S[i+1]=S[j-1]$;
				}
			}
		}
		\For{$len=n \to 1$}{
			\For{$row=1 \to n-len+1$}{
				\If{$dp[row][len+row-1]$}{\textbf{return} $len$;}
			}
		}		
	\end{algorithm}
\end{solution}

\item 有n个圆盘,从小到大编号为$1,2,3,\ldots,n$,随意放在一根柱子上。允许的操作是可以一次拿起最上方$k$个盘子,并将它们整体翻转放回柱子,设计一个操作方案将这些盘子按照从小到大的顺序放在柱子上。

\begin{solution}
	该问题可以转化为将数字序列$1,2,3,\ldots,n$由乱序状态,只经过反转操作恢复由小到大的顺序。思路是遍历找到每个数字,通过反转将它置于正确的位置上。伪代码如下:
	\begin{algorithm}[H]
		\SetAlgoLined
		\caption{$invert(end)$}
		\KwIn{待反转序列S,反转结束的序号end}
		\KwOut{反转得到的序列S}
		\For{$i=1$ to $end$}{
			$temp[i-start]=S[end-i+start]$;
		}
		\For{$i=1$ to $end$}{
			$S[i]=temp[i-start]$;
		}
	\end{algorithm}
	\begin{algorithm}[H]
		\SetAlgoLined
		\caption{$solve()$}
		\KwIn{序列S}
		\KwOut{反转序列S}
		\For{$i=n \to 2$}{
			\ForEach{$j=1 \to i$}{
				\If{$S[j]=i$ and $i \neq j$}{
					$invert(j)$;\\
					$invert(i)$;	
				}
			}
		}
	\end{algorithm}
\end{solution}

\item 给出建堆两种方法的伪代码
\begin{solution}
	以最大堆为例:$up2down()$为自顶向下建堆,$down2up()$为自底向上建堆。
	\begin{algorithm}
		\SetAlgoLined
		\caption{$up2down()$}
		\KwIn{待建堆数组A,长度为n}
		\KwOut{建堆后数组A,节点$i$的儿子为$2i$和$2i+1$}
		\For{$i=2 \to n$}{
			\While{$\lfloor i/2\rfloor>0$ and $A[\lfloor i/2\rfloor]<A[i]$}{
				$temp=A[\lfloor i/2\rfloor]$;\\
				$A[\lfloor i/2\rfloor]=A[i]$;\\
				$A[i]=temp$;\\
				$i=\lfloor i/2\rfloor$;
			}
		}
	\end{algorithm}

	\begin{algorithm}
		\SetAlgoLined
		\caption{$down2up()$}
		\KwIn{待建堆数组A,长度为n}
		\KwOut{建堆后数组A,节点$i$的儿子为$2i$和$2i+1$}
		\For{$i=\lfloor n/2 \rfloor \to 1$}{
			\While{$2i+1\leq n$}{
				\If{$A[2i+1]>A[2i]$}{
					\If{$A[2i+1]>A[i]$}{
						$temp=A[2i+1]$;\\
						$A[2i+1]=A[i]$;\\
						$A[i]=temp$;\\
						$i=2i+1$;
					}
				}
				\Else{
					\If{$A[2i]>A[i]$}{
						$temp=A[2i]$;\\
						$A[2i]=A[i]$;\\
						$A[i]=temp$;\\
						$i=2i$;
					}
				}
			}
			\If{$2i==n$}{
				\If{$A[n]>A[i]$}{
					$temp=A[i]$;\\
					$A[i]=A[n]$;\\
					$A[n]=temp$;
				}
			}			
		}
	\end{algorithm}
\end{solution}
\end{enumerate}

%========================================================================
\end{document}
