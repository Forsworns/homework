\documentclass[12pt,a4paper]{article}
\usepackage{ctex}
\usepackage{amsmath,amscd,amsbsy,amssymb,latexsym,url,bm,amsthm}
\usepackage{epsfig,graphicx,subfigure}
\usepackage{enumitem,balance}
\usepackage{wrapfig}
\usepackage{mathrsfs,euscript}
\usepackage[usenames]{xcolor}
\usepackage{hyperref}
\usepackage[vlined,ruled,commentsnumbered,linesnumbered]{algorithm2e}

\newtheorem{theorem}{Theorem}
\newtheorem{lemma}[theorem]{Lemma}
\newtheorem{proposition}[theorem]{Proposition}
\newtheorem{corollary}[theorem]{Corollary}
\newtheorem{exercise}{Exercise}
\newtheorem*{solution}{Solution}
\newtheorem{definition}{Definition}
\theoremstyle{definition}


%\numberwithin{equation}{section}
%\numberwithin{figure}{section}

\renewcommand{\thefootnote}{\fnsymbol{footnote}}

\newcommand{\postscript}[2]
{\setlength{\epsfxsize}{#2\hsize}
	\centerline{\epsfbox{#1}}}

\renewcommand{\baselinestretch}{1.0}

\setlength{\oddsidemargin}{-0.365in}
\setlength{\evensidemargin}{-0.365in}
\setlength{\topmargin}{-0.3in}
\setlength{\headheight}{0in}
\setlength{\headsep}{0in}
\setlength{\textheight}{10.1in}
\setlength{\textwidth}{7in}
\makeatletter \renewenvironment{proof}[1][Proof] {\par\pushQED{\qed}\normalfont\topsep6\p@\@plus6\p@\relax\trivlist\item[\hskip\labelsep\bfseries#1\@addpunct{.}]\ignorespaces}{\popQED\endtrivlist\@endpefalse} \makeatother
\makeatletter
\renewenvironment{solution}[1][Solution] {\par\pushQED{\qed}\normalfont\topsep6\p@\@plus6\p@\relax\trivlist\item[\hskip\labelsep\bfseries#1\@addpunct{.}]\ignorespaces}{\popQED\endtrivlist\@endpefalse} \makeatother

\usepackage{listings}

\begin{document}
\noindent

%========================================================================
\noindent\framebox[\linewidth]{\shortstack[c]{
\Large{\textbf{Homework09}}\vspace{1mm}\\
CS214-Algorithm and Complexity, Spring 2018}}
\begin{center}

\footnotesize{\color{black} Name: 杨培灏  \quad Student ID: 516021910233}
\end{center}

\begin{enumerate}

\item 如何将简单多边形划分为梯形(某些梯形可能退还为三角形)?给出算法及时间复杂性。
\begin{solution}
	设输入为有序的多边形顶点坐标序列,这里的有序是指从一点出发,作为一个有向图顺次连接成一个简单多边形。显然输入必须保证这个条件,否则无法唯一地确定一个多边形,与条件矛盾。
	
	从起始点开始顺次读取顶角。读入下一个顶角,计算当前顶角和前一个顶角的大小之和,如果小于$180^{\circ}$,做平行线与多边形边界交于一点,连接后成为一个梯形,否则向后顺次读取顶角。若仅剩余一个三角形无法分割则停止。当分割出平行四边形后,任意连接一条对角线化为两个三角形。
	\begin{center}
		\begin{algorithm}[H]
			\caption{divide()}
			\KwIn{多边形顶点序列$D=\{d_1,d_2,\ldots,d_n,d_1\}$}
			当前顶点$now=d_2$;\\
			上一个顶点$prev=d_1$
			\While{$now\neq d_1$}{
				\If{$now.degree+prev.degree>180^{\circ}$}{
					在now角作prev角的非两角公共边的平行线,交多边形为p点;\\
					划分多边形;\\
					\If{被划分区域为平行四边形}{
						分割为两个三角形;\\
					}
					\ElseIf{被划分区域为多边形}{
						调用该方法再次进行分割;\\
					}
					prev=p;\\
					now为按多边形边序列的方向转向的下一点;\\
				}\ElseIf{$now.degree+prev.degree<180^{\circ}$}{
					直接跳过;\\
				}\Else{
					在顶角prev与now较长的一边上截取一段小于对边的线段作为分割点p的位置;//等于时为平行四边形\\
					划分出一块梯形;\\
					prev=p;\\
					now为按多边形边序列的方向转向的下一点;\\
				}
			}
			
		\end{algorithm}
	\end{center}
	\textbf{复杂度分析:}该算法遍历了每个顶点,分割过程为常数时间操作,所以时间复杂度为$\mathcal{O}(n)$
\end{solution}

\item 给定平面上n条线段,设计算法用O(nlogn)时间确定其中是否有两条线段相交。
\begin{solution}
	对于给定的n条线段,先对他们的端点根据x坐标排序。找到x坐标最小的极端点作为起始点。设定一条扫描线,从左往右扫描(即随时间改变x坐标),扫描的状态集合S,是一个动态集合,描述扫描线的状态,包含当前与扫描线交叉的线段,这些线段按照交点的y坐标排序。该顺序动态变化,会在以下情况发生改变:遇到新的线段、现有线段结束。事件点为线段端点,并按之前得到的x坐标排序后的顺序依次对这些点进行处理。对于线段s的左端点:将线段s加入动态集合中、检查s在S中的相邻线段与s是否有交点;对于线段s的右端点:从动态集合S中删除线段,检查s在S中的相邻线段之间是否有交点。
	\begin{center}
		\begin{algorithm}[H]
			\caption{checkIntersection()}
			\KwIn{一个线段的集合$L=\{l_1,l_2,\ldots,l_n\}$}
			\KwOut{是否有交点}
			根据x坐标值以递增的顺序排序线段端点,把它们插入到最小堆E之中。
			\While{堆E是非空的}{
				从最小堆E中弹出一个点p;\\
				\If{p是一个左端点}{
					设l是左端点为p的线段;\\
					insert(l,S);\\
					$l_1\leftarrow above(l,S)$;//$l_1$为l上方邻近线段;\\
					$l_2\leftarrow below(l,S)$;//$l_2$为l下方邻近线段;\\
					\If{l与$l_1$相交在$q_1$}{
						输出存在交点;\\
						终止;\\
					}
					\If{l与$l_2$相交在$q_2$}{
						输出存在交点;\\
						终止;\\
					}
				}\ElseIf{p是一个右端点}{
					设l是右端点为p的线段;\\
					insert(l,S);\\
					$l_1\leftarrow above(l,S)$;//$l_1$为l上方邻近线段;\\
					$l_2\leftarrow below(l,S)$;//$l_2$为l下方邻近线段;\\
					\If{$l_1$与$l_2$相交在$q$}{
						输出存在交点;\\
						终止;\\
					}
				}
			}
			输出不存在交点;\\
			终止;
		\end{algorithm}
	\end{center}
	\textbf{复杂度分析:}该算法首先对所有顶点进行排序,该步的时间复杂度为$\mathcal{O}(n\log n)$。之后扫描线遍历时,时间复杂度与交点数有关,为$\mathcal{O}((2n+m)\log(2n+m))$,而检查状态集中的线段是否是相交的是一个常数时间的操作。所以总的时间复杂度为$\mathcal{O}((2n+m)\log(2n+m))$。事实上这个算法中,一旦找到一个交点,已经可以得出结论,算法就会结束,所以时间复杂度是$\mathcal{O}(n\log n)$。
\end{solution}

\item 如何确定一个给定的多边形是简单多边形
\begin{solution}
	对多边形顶点根据x坐标值排序,之后同样找到x坐标最小的极端点作为起始点。设定一条扫描线,从左往右扫描(即随时间改变x坐标),扫描的状态集合S,是一个动态集合,描述扫描线的状态,包含当前与扫描线交叉的线段,这些线段按照交点的y坐标排序。该顺序动态变化,会在以下情况发生改变:遇到新的线段、现有线段结束,遇到两条线段交点。事件点为线段端点,并按之前得到的x坐标排序后的顺序依次对这些点进行处理。对于线段s的左端点:将线段s加入动态集合中、检查s在S中的相邻线段与s是否有交点;对于线段s的右端点:从动态集合S中删除线段,检查s在S中的相邻线段之间是否有交点;对于线段s和t的交点:交换s和t的次序,检查s、t和他们的相邻线段是否有交点。
	
	如果这n条线段的交点数为n,则说明除了顶点外,没有其他交点,是一个简单多边形,否则不是简单多边形。
	\begin{center}
		\begin{algorithm}[H]
			\caption{intersection()}
			\KwIn{一个线段的集合$L=\{l_1,l_2,\ldots,l_n\}$}
			\KwOut{交点数m}
			根据x坐标值以递增的顺序排序线段端点,把它们插入到最小堆E之中。
			\While{堆E是非空的}{
				从最小堆E中弹出一个点p;\\
				\If{p是一个左端点}{
					设l是左端点为p的线段;\\
					insert(l,S);\\
					$l_1\leftarrow above(l,S)$;//$l_1$为l上方邻近线段;\\
					$l_2\leftarrow below(l,S)$;//$l_2$为l下方邻近线段;\\
					\If{l与$l_1$相交在$q_1$}{
						在S中交换l和$l_1$;\\
						$q_1$入堆;\\
					}
					\If{l与$l_2$相交在$q_2$}{
						在S中交换l和$l_2$;\\
						$q_2$入堆;\\
					}
				}\ElseIf{p是一个右端点}{
					设l是右端点为p的线段;\\
					insert(l,S);\\
					$l_1\leftarrow above(l,S)$;//$l_1$为l上方邻近线段;\\
					$l_2\leftarrow below(l,S)$;//$l_2$为l下方邻近线段;\\
					\If{$l_1$与$l_2$相交在$q$}{
						在S中交换$l_1$和$l_2$;\\
						q入堆;\\
					}
				}\Else{
					//p是交点\\
					设相交于p点的线段是$l_1$和$l_2$,且$l_1$在上方;\\
					$l_3\leftarrow above(l_1,S)$;//$l_3$为$l_1$上方邻近线段;\\
					$l_4\leftarrow below(l_2,S)$;//$l_4$为$l_2$下方邻近线段;\\
					\If{$l_2$与$l_3$相交在$q_1$}{
						在S中交换$l_2$和$l_3$;\\
						$q_1$入堆;\\
					}
					\If{$l_1$与$l_4$相交在$q_2$}{
						在S中交换$l_1$和$l_4$;\\
						$q_2$入堆;\\
					}
				}
			}
		\end{algorithm}
	\end{center}
	
	
	\textbf{复杂度分析:}该算法首先对所有顶点进行排序,该步的时间复杂度为$\mathcal{O}(n\log n)$。之后扫描线遍历时,时间复杂度与交点数有关,为$\mathcal{O}((2n+m)\log(2n+m))$,而检查状态集中的线段是否是相交的是一个常数时间的操作。所以总的时间复杂度为$\mathcal{O}((2n+m)\log(2n+m))$。事实上这个算法中,一旦m超过n,已经可以得出结论,算法就会结束,所以时间复杂度是$\mathcal{O}(n\log n)$。
\end{solution}

\item 设P和Q是两个简单多边形,顶点总数为n,设计$\mathcal{O}(n\log n)$时间算法确定P和Q是否相交
\begin{solution}
	首先对输入的两个多边形的顶点序列排序,选择x坐标最小的极端点作为扫描线的起始位置。
	
	由于已知了P和Q是简单多边形,所以两多边形的边自身除了顶点外不会有其他交点。设定扫描线从左往右,设两个状态集S和T,分别记录当前与扫描线相交的P和Q中的线段,只需要检查S和T中的线段是否相交。
	
	或是仿照上例,直接统计这n条线段的交点数m是否为n,如果是n,则说明除了两个简单多边形的顶点以外他们没有其余交点,即多边形P、Q间没有交集,否则P、Q相交。
	
	该算法的时间复杂度为$\mathcal{O}((2n+m)\log(2n+m))$。同上一题一样,一旦m超过n,已经可以得出结论,算法就会结束,所以时间复杂度是$\mathcal{O}(n\log n)$。
\end{solution}

\item 矩阵$A_{m\times n}$和矩阵$B_{n\times p}$相乘得到矩阵$C_{m\times p}$,根据矩阵相乘的定义,矩阵$C$中的元素为:
$$C_{ij}=\sum_{k=1}^n{A_{ik}B_{kj}}
$$
根据这一定义,为求矩阵$C$,共需$n^3$次乘法。那么$n$个矩阵$M_1,M_2,\ldots,M_n$相乘需要多少次乘法呢?由于矩阵乘法满足分配律,因此多个矩阵相乘时有多种计算方式,而每种方式所需的计算量是不同的。假设$M_1$的维数是$2\times10$,$M_2$的维数是$10\times2$,$M_3$的维数为$2\times10$,则计算$M_1*M_2*M_3$有两种方式:
$$(M_1*M_2)*M_3$$ 
$$M_1*(M_2*M_3)$$
这两种计算方式得到的结果矩阵是一样的,但所需乘法数量分别为80和400!
\begin{solution}
	两个相乘的矩阵大小分别设为$i\times j$和$j\times k$,那么这两个矩阵相乘后,实际进行的乘法的数量为$i\times j\times k$。
	
	设总的乘法次数为$S$。考虑$M_1,M_2,\ldots,M_n$,它们的规模分别是$(i_1,i_2),(i_2,i_3),\ldots,(i_{n-1},i_n)$。构造一序列$I=\{i_1,\ldots,i_n\}$。运用结合律后,假设我们计算的是$M_{j-1}$和$M_{j}$,进行了$i_{j-1}\times i_j\times i_{j+1}$步乘法。这步的结果是,我们相当于移除了序列I中的$i_j$,新得到了$I'=\{i_1,\ldots,i_{j-1},i_{j+1},\ldots,i_n\}$。显然,当I中只剩下两个数字的时候,即计算出最终结果的时候,算法结束(事实上$i_1$和$i_n$是不可能被移除的两个数字,因为乘积的规模是确定的)。每次移除$i_j$,总的乘法次数$S$的增量值为$i_j$元素时其与左右元素的乘积。
	
	只要我们每次从$I$中移除的是最大的元素,就可以保证最后的总的乘法次数$S$最小。该算法的正确性在于,一个数被计算到S的增量中,当仅当它被移除或它的邻居被移除。而总的移除次数是固定的,所以优先移除最大的元素可以保证它被计算的次数尽可能得少。被移除的顺序也就对应决定了矩阵乘积的结合顺序。
	
	所以算法是
	\begin{algorithm}
		\caption{minMulTimes()}
		\KwIn{矩阵$M_1,M_2,\ldots,M_n$}
		\KwOut{乘积总次数S}
		遍历一次矩阵序列构造出$I=\{i_1,\ldots,i_n\}$;
		\While{$I.size()>2$}{
			从$i_2$到$i_{n-1}$中找到最大值;\\
			$i_k$,$i_l$是$i_j$左右的元素;\\
			$S=S+i_k\times i_j\times i_l$;\\
			从$I$中移除$i_j$;
		}
	\end{algorithm}

	我们可以预先为I中的元素排序,建一个表保存这个顺序。那么在挑选最大值时就可以直接根据这个表挑选,算法的时间复杂度为$\mathcal{O}(n\log n)$。
\end{solution} 

\item 有$n$种液体$S_1,S_2,\ldots,S_n$,都含有$A$,$B$两种成分,含量分别为$\{a_1,b_1\},\{a_2,b_2\},\ldots,\{a_n,b_n\}$,$a_i+b_i<100\%$。现欲利用这$n$种液体配制目标液体T,使之$A$和$B$的含量分别为$x$和$y$。设计算法判别能否成功配制,并给出算法时间复杂性。

\begin{solution}
	设饮料的体积分别为:$V_1,V_2,\ldots,V_n$。则等式为
	$$\sum_{i=1}^{n}{a_iV_i}=x\sum_{i=1}^{n}{V_i}$$
	$$\sum_{i=1}^{n}{b_iV_i}=y\sum_{i=1}^{n}{V_i}$$
	结合两式,可得只要满足下面这个等式即可
	$$\sum_{i=1}^{n}{(a_i-\frac{x}{y}b_i)V_i}=0$$
	该等式中各项的体积$V_i$的各项系数已知。由于$V_i\geq0$,故若$\forall i\in\{1,\ldots,n\}$,$a_i-\frac{x}{y}b_i>0$或$\forall i\in\{1,\ldots,n\}$,$a_i-\frac{x}{y}b_i>0$,则必然无解。否则存在无数多组解。
	
	所以判别算法为,$i=1\to n$,分别计算$a_i-\frac{x}{y}b_i$,如果不是均大于0或均小于0,就必然存在解。这个算法只需要遍历一次输入,所以它的时间复杂度为$\mathcal{O}(n)$。
\end{solution}
\end{enumerate}

%========================================================================
\end{document}
