\documentclass[12pt,a4paper]{article}
\usepackage{ctex}
\usepackage{amsmath,amscd,amsbsy,amssymb,latexsym,url,bm,amsthm}
\usepackage{epsfig,graphicx,subfigure}
\usepackage{enumitem,balance}
\usepackage{wrapfig}
\usepackage{mathrsfs,euscript}
\usepackage[usenames]{xcolor}
\usepackage{hyperref}
\usepackage[vlined,ruled,commentsnumbered,linesnumbered]{algorithm2e}

\newtheorem{theorem}{Theorem}
\newtheorem{lemma}[theorem]{Lemma}
\newtheorem{proposition}[theorem]{Proposition}
\newtheorem{corollary}[theorem]{Corollary}
\newtheorem{exercise}{Exercise}
\newtheorem*{solution}{Solution}
\newtheorem{definition}{Definition}
\theoremstyle{definition}


%\numberwithin{equation}{section}
%\numberwithin{figure}{section}

\renewcommand{\thefootnote}{\fnsymbol{footnote}}

\newcommand{\postscript}[2]
{\setlength{\epsfxsize}{#2\hsize}
	\centerline{\epsfbox{#1}}}

\renewcommand{\baselinestretch}{1.0}

\setlength{\oddsidemargin}{-0.365in}
\setlength{\evensidemargin}{-0.365in}
\setlength{\topmargin}{-0.3in}
\setlength{\headheight}{0in}
\setlength{\headsep}{0in}
\setlength{\textheight}{10.1in}
\setlength{\textwidth}{7in}
\makeatletter \renewenvironment{proof}[1][Proof] {\par\pushQED{\qed}\normalfont\topsep6\p@\@plus6\p@\relax\trivlist\item[\hskip\labelsep\bfseries#1\@addpunct{.}]\ignorespaces}{\popQED\endtrivlist\@endpefalse} \makeatother
\makeatletter
\renewenvironment{solution}[1][Solution] {\par\pushQED{\qed}\normalfont\topsep6\p@\@plus6\p@\relax\trivlist\item[\hskip\labelsep\bfseries#1\@addpunct{.}]\ignorespaces}{\popQED\endtrivlist\@endpefalse} \makeatother

\usepackage{listings}

\begin{document}
\noindent

%========================================================================
\noindent\framebox[\linewidth]{\shortstack[c]{
\Large{\textbf{Homework05}}\vspace{1mm}\\
CS214-Algorithm and Complexity, Spring 2018}}
\begin{center}

\footnotesize{\color{black} Name: 杨培灏  \quad Student ID: 516021910233}
\end{center}

\begin{enumerate}

\item 1 给定测度空间中位于同一平面的$n$个点,已知任意两点之间的距离$d_{ij}$,存储在矩阵$D$中,最大的$d_{ij}$被称为图的直径。
该问题的直观解法就是把$D$扫描一遍,选择其中最大的元素即可。由于是在一个测度空间中,因此dij满足距离的基本要求,即非负性、对称性和三角不等式,我们就可以给出一种时间亚线性的近似算法。算法很简单,由原来确定性算法的检查整个矩阵改为只随机检查$D$的某一行,这样时间复杂性就由原来的$\mathcal{O}(n^2)$减少为$\mathcal{O}(n)$,相对于输入规模$n^2$而言,这是一个时间亚线性的算法。那么时间代价减小的同时,证明解不会小于最优值的一半

\begin{proof}
	由于我们从图中一点出发,遍历了与其它(n-1)个点的距离,不妨以矩阵$D$的第一行为例。那么无非会有以下两种情况:
	\begin{enumerate}
		\item 直径在$d_{12},d_{13},\ldots,d_{1n}$中,那么遍历该行,显然找到了最优解。
		\item 直径不在该行,首先我们知道该距离具有对称性$d_{1i}=d_{i1}$。那么对于该行任意两个元素$d_{1i}$和$d_{1j}$,点i和点j间的距离为$d_{ij}$,就覆盖了第一行外所有的点间的距离。又因为在同一测度空间,由三角不等式,有$d_{1i}+d_{1j}>d_{ij}$,所以$d_{1i}$和$d_{1j}$不可能均小于$d_{ij}/2$。由$i$、$j$的任意性,我们知$d_{1i}$中的最大值不可能小于最优值的一半。
	\end{enumerate}
	综上所述,解不会小于最优值的一半。
\end{proof}

\item $G=(V,E)$是一个无向图,每个顶点的度数都为偶数,设计线性时间算法,给$G$中每条边一个方向,使每个顶点的入度等于出度。(请先简单说明算法思想,再给出伪代码,然后证明其时间复杂性符合要求)

\begin{solution}
	由于该图的连通性并不影响度数的性质,可以当做多个相同性质的无向图处理。所以下面只讨论该图为连通图的情况。那么由于无向图$G$中每个顶点的度数均为偶数,该图存在欧拉回路。
	那么接下来可以任意取一点,开始遍历经过该点的欧拉回路,遍历过程中,经过回路的方向就是每条边的方向,直到我们遍历完图G所有的边,时间复杂度为$m\mathcal{O}(|E|)$。
	所以问题就转化成了如何找到欧拉回路:
	\begin{algorithm}
		\caption{euler(vertex)}
		\KwIn{当前点vertex}
		\KwOut{欧拉回路标记}
		\For{neighbor in vertex的所有邻接点}{
			\If{该边(vertex,neighbor)未被标记过}{
				标记该边方向;\\
				euler(neighbor);
			}
		}
	\end{algorithm}
	该求解欧拉回路的算法的时间复杂度为$m\mathcal{O}(|E|)$。因为对于每个端点,都会遍历它相关联的边,基本语句为条件判断部分,那么会判断$2|E|$次。事实上在构造欧拉回路的过程中,在标记边的时候,就确定了题目中要求的各边方向。
\end{solution}

\end{enumerate}

%========================================================================
\end{document}
