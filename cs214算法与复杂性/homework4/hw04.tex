\documentclass[12pt,a4paper]{article}
\usepackage{ctex}
\usepackage{amsmath,amscd,amsbsy,amssymb,latexsym,url,bm,amsthm}
\usepackage{epsfig,graphicx,subfigure}
\usepackage{enumitem,balance}
\usepackage{wrapfig}
\usepackage{mathrsfs,euscript}
\usepackage[usenames]{xcolor}
\usepackage{hyperref}
\usepackage[vlined,ruled,commentsnumbered,linesnumbered]{algorithm2e}
\renewcommand{\algorithmcfname}{算法}
\newtheorem{theorem}{Theorem}
\newtheorem{lemma}[theorem]{Lemma}
\newtheorem{proposition}[theorem]{Proposition}
\newtheorem{corollary}[theorem]{Corollary}
\newtheorem{exercise}{Exercise}
\newtheorem*{solution}{Solution}
\newtheorem{definition}{Definition}
\theoremstyle{definition}


%\numberwithin{equation}{section}
%\numberwithin{figure}{section}

\renewcommand{\thefootnote}{\fnsymbol{footnote}}

\newcommand{\postscript}[2]
{\setlength{\epsfxsize}{#2\hsize}
	\centerline{\epsfbox{#1}}}

\renewcommand{\baselinestretch}{1.0}

\setlength{\oddsidemargin}{-0.365in}
\setlength{\evensidemargin}{-0.365in}
\setlength{\topmargin}{-0.3in}
\setlength{\headheight}{0in}
\setlength{\headsep}{0in}
\setlength{\textheight}{10.1in}
\setlength{\textwidth}{7in}
\makeatletter \renewenvironment{proof}[1][Proof] {\par\pushQED{\qed}\normalfont\topsep6\p@\@plus6\p@\relax\trivlist\item[\hskip\labelsep\bfseries#1\@addpunct{.}]\ignorespaces}{\popQED\endtrivlist\@endpefalse} \makeatother
\makeatletter
\renewenvironment{solution}[1][Solution] {\par\pushQED{\qed}\normalfont\topsep6\p@\@plus6\p@\relax\trivlist\item[\hskip\labelsep\bfseries#1\@addpunct{.}]\ignorespaces}{\popQED\endtrivlist\@endpefalse} \makeatother

\usepackage{listings}
\lstset{
	tabsize=2 % sets default tabsize to 2 spaces
}
\begin{document}
\noindent

%========================================================================
\noindent\framebox[\linewidth]{\shortstack[c]{
\Large{\textbf{Homework04}}\vspace{1mm}\\
CS214-Algorithm and Complexity, Spring 2018}}
\begin{center}

\footnotesize{\color{black} Name: 杨培灏  \quad Student ID: 516021910233}
\end{center}

\begin{enumerate}

\item 背包问题:假设有$n$个不同的物品,被分成了$k$组,每组物品只能选一个(可以不选),那么背包是否能够恰好放满?
\begin{solution}
	数组dp($k\times max\_capcity$)表示到第$k$组中物品为止可以填满的背包空间大小。初始$dp[1][w_1^j]=true,w_1^j \in W_1$,其余均为false。
	\begin{algorithm}
		\caption{$backpack()$}
		\KwIn{n个物品重量$w_i$(按组分为$W_1,W_2,\ldots,W_k$)}
		\KwOut{布尔变量表示能否恰好放满}
		\For{$W_i$ in $\{W_2,\ldots,W_k\}$}
		{
			\For{$w$ in $1$ to $max\_capcity$}{
				\For{$w_i^j$ in $W_i$}{
					\If{$w-w_i^j>0$}{
						$dp[i][w]=dp[i-1][w]$ or $dp[i-1][w-w_i^j]$;
					}
				}
			}
		}
	\end{algorithm}
\end{solution}
\item 证明二分查找是有序序列查找的最优算法
\begin{solution}
	建立类似排序的决策树,设输入序列为$x_1,x_2,\ldots,x_n$,每个非叶节点表示对输入序列的一次查询,查询结果有两种可能,根据查询结果可以选择左右分支。直到查询到叶子节点。叶子节点表示所有可能的输出,即序列中的某个元素或空串,所以共有$(n+1)$种情况。又因为该树为二叉树,所以它的高度最大为$\log n$。而最坏情况下,算法的运行时间为决策树的高度。因此不可能存在一个查找算法的时间复杂度小于$\mathcal{O}(\log n)$。也即二分查找算法是有序序列查找的最优算法。
\end{solution}
\item 如何修改KMP算法,使之能够获得字符串$B$在字符串$A$中出现的次数
\begin{solution}
	\begin{algorithm}
		\caption{$String_Match(A,n,B,m)$}
		\KwIn{A(长度为n的字符串)和B(长度为m的字符串),数组next}
		\KwOut{count(字符串B在字符串A中出现的次数)}
		j=1;
		i=1;
		count=0;
		\While{$i\leq n$}{
			\If{B[j]=A[i]}{
				j=j+1;\\
				i=i+1;
			}
			\Else{
				j=next[j]+1;\\
				\If{j=0}{
					j=1;\\
					i=i+1;
				}
				\If{j=m+1}{
					counter=counter+1;\\
					j=1;\\
					i=i+1;
				}
			}
		}
	\end{algorithm}
	\begin{algorithm}
		\caption{$Compute_Next(B,m)$}
		\KwIn{B(长度为m的字符串)}
		\KwOut{next(大小为m的数组)}
		next[1]=-1;\\
		next[2]=0;\\
		\For{i=3 to n}{
			j=next[i-1]+1;\\
			\While{$b_{i-1}\neq b_j$ and $j>0$}{
				j=next[j]+1;
			}
			next[i]=j;
		}
	\end{algorithm}
\end{solution}

\item 序列T和P的最长公共子序列(LCS)L定义为T和P的共同子序列中最长的一个,而最短公共超序列(SCS)S定义为所有以T和P为子序列的序列中最短的一个。设计高效算法求序列T和P的LCS和SCS的长度。
\begin{solution}
	SCS的长度应该是$|T|+|P|-LCS$,即以T和P的最长公共子序列为基础,增加上两者不重合的部分。
	而LCS的求法如下,利用动态规划的思想:
	\begin{algorithm}
		\caption{LCS(T, P)}
		\KwIn{序列T,P}
		\KwOut{最长子序列长度}
		设m、n分别为T、P的长度;
		初始化二维数组dp为$(m+1)\times (n+1)$的零矩阵,dp[i][j]表示序列T[i]和P[j]结尾的最长公共子序列长度;
		\For{$i$ in $1\to m$}{//0行和0列无意义存为0,只用来简化对下标的讨论
			\For{$j$ in $1\to n$}{
				\If{T[i]=P[j]}{
					dp[i][j]=dp[i-1][j-1]+1;
				}else{
					dp[i][j]=0;
				}
			}
		}
		最后遍历输出dp中元素的最大值即为LCS值
	\end{algorithm}
\end{solution}

\item 设计高效算法求序列中出现次数大于$n/4$的所有元素
\begin{solution}
	具体思路和计算次数大于$n/2$的元素相同,因为出现次数大于$n/4$的所有元素最多可能有三个,那么每次删除四个元素,出现次数大于$n/4$的元素出现次数仍然大于$(n-4)/4$。这里构造三个可能的选项,然后如果新的数在三个选项中,那么该选项加一,否则所有选项均要减一,如果出现了为0的选项,将会删除掉它(用新的元素替换)。时间复杂度为$\mathcal{O}(n)$
	\begin{algorithm}
		\caption{findMajorityOf4(array X)}
		\KwIn{X(一个大小为n的正数序列)}
		\KwOut{Majority[3]如果不存在则为-1}
		C[1]=X[1];
		C[2]=X[2];
		M[1]=1;
		M[2]=1;
		pos=2;
		counter=0;\\
		\While{C[2]=C[1]}{
			pos=pos+1;\\
			C[2]=X[pos];\\
			M[1]=M[1]+1;
		}
		\While{C[3]=C[1] or C[3]=C[2]}{
			pos=pos+1;\\
			\If{C[3]=C[1]}{
				C[3]=X[pos];\\
				M[1]=M[1]+1;\\
			}\Else{
				C[3]=X[pos];\\
				M[2]=M[2]+1;
			}
		}
		\For{pos in pos $\to$ n}{
			\If{X[pos]=C[1]}{
				M[1]=M[1]+1;\\
			}\ElseIf{X[pos]=C[2]}{
				M[2]=M[2]+1;\\
			}\ElseIf{X[pos]=C[3]}{
				M[3]=M[3]+1;
			}
			\Else{
				M[1]=M[1]-1;\\
				M[2]=M[2]-1;\\
				M[3]=M[3]-1;\\
				\If{M[1]=0}{
					C[1]=X[pos];
					M[1]=1;	
				}\ElseIf{M[2]=0}{
					C[2]=X[pos];
					M[2]=1;	
				}\ElseIf{M[3]=0}{
					C[3]=X[pos];\\
					M[3]=1;	
				}
			}
		}
		M[1]=M[2]=M[3]=0;\\
		\For{i in $1\to n$}{
			\If{X[i]=C[1]}{
				M[1]=M[1]+1;
			}\ElseIf{X[i]=C[1]}{
				M[2]=M[2]+1;
			}\ElseIf{X[i]=C[1]}{
				M[3]=M[3]+1;
			}
		}
		\If{M[1]<n/4}{M[1]=-1;}
		\If{M[2]<n/4}{M[2]=-1;}
		\If{M[3]<n/4}{M[3]=-1;}
	\end{algorithm}
\end{solution}

\item 对于任意给定的4个$1\to 10$之间的整数(可以相同),判断是否可以通过整数四则运算得到24
\begin{solution}
	首先考虑将运算简化。可以发现,一个包含四个数的有序数列的运算只有两种基本结构(2,2)和(3,1),即两两先进行运算再对两个结果进行运算,或者三个数的运算得结果与另一个数运算。这两种情况有重合的部分,但是这里不做考虑。同时,由于四个数可以交换顺序,但是两个参与运算的数之间的数的顺序是没有意义的,那么(2,2)可以有$4\choose 2 $$/2 = 3$种,而(3,1)可以有$4\choose 1$$\times$$3\choose 1$。而每个运算符会有4种选择,共$4^3=64$种。即最坏情况下,该算法将检验960种子情况。
	\begin{algorithm}
		\caption{Cal(a,b,c,d)}
		\KwIn{输入参与运算的变量(2到4个)}
		\KwOut{输入变量数目为2或3时,经过运算的结果的集合;输入数量为4时,输出可行与否}
		\If{输入参数数量为4}{
			 \If{$24\in Cal(Cal(a,b,c),d)$}{return $True$;}
			 \If{$24\in Cal(Cal(a,b,d),c)$}{return $True$;}
			 \If{$24\in Cal(Cal(a,b),Cal(c,d))$}{return $True$;}
			 按照上面的描述,将各种有序数列及其运算写出。
		}
		\ElseIf{输入参数数量为3}{
			 return $\{Cal(Cal(a,b),c),\ldots ,Cal(Cal(a,c),b)\}$;
			 
		}\Else{
			 return $\{a+b,a-b,b-a,a*b,a/b,b/a\}$;	 
		}
	\end{algorithm}
\end{solution}

\item 如果序列中存在众元,设计分治法找出众元(这里的众元为出现次数超过n/2的元素)。
\begin{solution}
	如果众元存在,那么经过排序后,它一定为序列的中数,那么这里就是要用分治法去寻找中位数。我们首先将数组拆分成两部分,将他们分别排序,然后对这两个有序的子序列利用分治法求解中位数。
	\begin{algorithm}
		\caption{quickSort(A,low,high)}
		\KwIn{序列A}
		\KwOut{排序后的序列}
		\If{$low\ge high$}{
			return;
		}
		first=low;\\
		last=high;\\
		key=a[first];\\
		\While{$first<last$}{
			\While{$A[last]\geq key$ and $first<last$}{last=last-1;}
			A[first]=A[last];\\
			\While{$A[first]\leq key$ and $first<last$}{first=first+1;}
			A[last]=A[first];\\
		}
		A[first]=key;\\
		quickSort(A,low,first-1);\\
		quickSort(A,first+1,high);
	\end{algorithm}
\end{solution}

\end{enumerate}

%========================================================================
\end{document}
