\documentclass[12pt,a4paper]{article}
\usepackage{ctex}
\usepackage{amsmath,amscd,amsbsy,amssymb,latexsym,url,bm,amsthm}
\usepackage{epsfig,graphicx,subfigure}
\usepackage{enumitem,balance}
\usepackage{wrapfig}
\usepackage{mathrsfs,euscript}
\usepackage[usenames]{xcolor}
\usepackage{hyperref}
\usepackage[vlined,ruled,commentsnumbered,linesnumbered]{algorithm2e}

\newtheorem{theorem}{Theorem}
\newtheorem{lemma}[theorem]{Lemma}
\newtheorem{proposition}[theorem]{Proposition}
\newtheorem{corollary}[theorem]{Corollary}
\newtheorem{exercise}{Exercise}
\newtheorem*{solution}{Solution}
\newtheorem{definition}{Definition}
\theoremstyle{definition}


%\numberwithin{equation}{section}
%\numberwithin{figure}{section}

\renewcommand{\thefootnote}{\fnsymbol{footnote}}

\newcommand{\postscript}[2]
{\setlength{\epsfxsize}{#2\hsize}
	\centerline{\epsfbox{#1}}}

\renewcommand{\baselinestretch}{1.0}

\setlength{\oddsidemargin}{-0.365in}
\setlength{\evensidemargin}{-0.365in}
\setlength{\topmargin}{-0.3in}
\setlength{\headheight}{0in}
\setlength{\headsep}{0in}
\setlength{\textheight}{10.1in}
\setlength{\textwidth}{7in}
\makeatletter \renewenvironment{proof}[1][Proof] {\par\pushQED{\qed}\normalfont\topsep6\p@\@plus6\p@\relax\trivlist\item[\hskip\labelsep\bfseries#1\@addpunct{.}]\ignorespaces}{\popQED\endtrivlist\@endpefalse} \makeatother
\makeatletter
\renewenvironment{solution}[1][Solution] {\par\pushQED{\qed}\normalfont\topsep6\p@\@plus6\p@\relax\trivlist\item[\hskip\labelsep\bfseries#1\@addpunct{.}]\ignorespaces}{\popQED\endtrivlist\@endpefalse} \makeatother

\usepackage{listings}

\begin{document}
\noindent

%========================================================================
\noindent\framebox[\linewidth]{\shortstack[c]{
\Large{\textbf{Homework02}}\vspace{1mm}\\
CS214-Algorithm and Complexity, Spring 2018}}
\begin{center}

\footnotesize{\color{black} Name: 杨培灏  \quad Student ID: 516021910233}
\end{center}

\begin{enumerate}

\item 求解递推关系式$T(n)=n+\sum_{i=1}^{n-1} T(i)$,其中$T(1)=1$.

\begin{solution}
{\setlength\abovedisplayskip{0.7pt}
\setlength\belowdisplayskip{0.7pt}
$$T(n)=n+\sum_{i=1}^{n-1} T(i)$$}
{\setlength\abovedisplayskip{0.7pt}
\setlength\belowdisplayskip{0.7pt}
$$T(n-1)=n-1+\sum_{i=1}^{n-2} T(i)$$}
{\setlength\abovedisplayskip{0.7pt}
\setlength\belowdisplayskip{0.7pt}
$$T(n)=1+2T(i)$$}
{\setlength\abovedisplayskip{0.7pt}
\setlength\belowdisplayskip{0.7pt}
$$T(n)+1=2(T(n)+1)$$}
{\setlength\abovedisplayskip{0.7pt}
\setlength\belowdisplayskip{0.7pt}
$$T(n)=2^{n}-1$$}
\end{solution}

\item 在寻找一对一映射问题中,算法结束时集合S是否可能为空集?请证明你的结论。

\begin{proof}
在寻找一对一映射问题中,算法结束时集合S不可能为空集。

显然,如果出现环形的映射关系,则由于环内元素一一对应且映射关系闭合,组成环的元素均不可能被删除。

由题意,$a_1$必须映射到一个元素,且不可能是$a_1$自身否则无法被删除。那么假设集合中的元素$a_1$映射到了$a_2$,则$a_2$在$a_1$被删除前不会被删除。设$a_2$映射到了$a_3$(若$a2$映射到$a_1$或$a_2$,则同样不可能被删除掉),则$a_3$在$a_2$被删除前不会被删除。类似地重复该操作,可以找到一条映射关系组成的链条。但是由于集合内元素有限,该操作不可能无限进行下去。即必然会存在$a_j$会映射到之前的元素$a_i$, $i\in \{1,\ldots,j\}$。那么$a_i$到$a_j$形成了环形的映射关系,这些元素是不可能被删除的。
\end{proof}

\item 课上的最大连续子序列是指和最大,如果要求是积最大呢?设空序列的积为1,只需求出最大值即可,不需要知道是哪个序列

\begin{solution}
	首先定义max()返回括号内各个参数的最大值,min()返回括号内各个参数的最小值。
	
	如果是求积最大,设$a_i$为序列的第$i$个元素。设$max_{i}$为长度为$i$的最大连续子序列积,$min_{i}$为长度为$i$的最小连续子序列积,$tailmin_{i}$为长度为$i$的序列尾部最小连续子序列积,$tailmax_{i}$为长度为$i$的序列尾部的最大连续子序列积。

	首先考虑最基础的情况,当序列长度为1时,$max_{1}=max(a_1,1),min_1=min(a_1,1)$。
	
	如果已知长度为$k$的序列的$max_{k},tail_{k}$,那么对于长度为$(n+1)$的序列:
	
	如果$max_k$和$min_k$的序列最后一项均为$a_k$,只需判断是否需要接上$a_{k+1}$,则$$max_{k+1}=max(max_k,max_{k}*a_{k+1}),min_{k}*a_{k+1})$$ $$min_{k+1}=min(min_k,max_{k}*a_{k+1},min_{k}*a_{k+1})$$
	
	如果$max_k$和$min_k$的序列最后一项均不为$a_k$,则$$tailmax_{k+1}=max(1,a_{k+1},tailmax_k*a_{k+1},tailmin_{k}*a_{k+1})$$ $$tailmin_{k+1}=min(1,a_{k+1},tailmax_{k}*a_{k+1},tailmin_{k}*a_{k+1})$$ $$max_{k+1}=max(max_k,tailmax_{k+1})$$ $$min_{k+1}=min(min_k,tailmin_{k+1})$$
	
	如果$max_k$和$min_k$中一个最后一项均不为$a_k$,仿照上面的两种情况可以得出递推关系,这里不再赘述。
	
	由归纳基础及递推关系可知最大子序列之积。
\end{solution}

\item 背包问题中如果各种大小的物品的数量不限,那么如何知道背包是否能够恰好放满?

\begin{solution}
	记背包容量为$vmax$,$n$种物品的大小从小到大排序分别为$w_i,i\in\{1,\ldots,n\}$,构成集合$Weights$。包内空间占用为$v[i],i\in\{1,\ldots,vmax\}$,如果可能出现这种情况则置为1,不可能存在这种情况置为0。
	
	首先,对$\forall i\in \{1,\ldots\,n\},v[w_i]=1$,之后伪代码为:
	\begin{lstlisting}
	for(size in (w_1, ..., vmax)) {
	//对每个容量都要考察
		if(v[size]) continue; 
		else {
			for(j in Weights){
			//检查到某物品使该情况可能发生
			if((size-j) > 0){
			//最初可能考察的容量比较小
				v[size] = v[size-j];
				if(v[size]) break;
				//某容量可行,不必循环
			} 
			else break;
			}
		}
	}
	\end{lstlisting}
	最后检查v[m]是否为1,即可知晓是否可以塞满背包。
\end{solution}

\item 写出轮廓问题的伪代码

\begin{solution}
	设含有n个元素的数组buidings表示n个建筑的轮廓线,为全局变量
	\begin{algorithm}
		\caption{divideBuildings(startIndex,endIndex)}
		\KwIn{轮廓线序号起点startIndex,终点endIndex}
		\KwResult{合并后的轮廓线outline}
		\eIf{startIndex==endIndex}{
			return buildings[startIndex];
		}{}
		\eIf{startIndex==endIndex-1}{
			return merge(building[startIndex], building[endIndex]);
		}
		{
			midIndex = startIndex + endIndex;
			outline1 = divideBuildings(startIndex, midIndex);
			outline2 = divideBuildings(midIndex+1, endIndex);
			return merge(outline1, outline2);
		}
	\end{algorithm}
	\begin{algorithm}
		\SetAlgoLined
		\caption{merge(outline1,outline2)}
		\KwIn{两个轮廓线outline1,outline2}
		\KwOut{合并后的轮廓线outline}
		$i = 0, j = 0, k = 0$;
		outline[];
		\While{$i<size(outline1)$ \textbf{and} $j<size(outline2)$}{
			\eIf{$outline1[i]<ouline2[j]$}{
				$outline[k] = outline1[i]$;
				$k = k + 1$;
				$outline[k] = max(outline1[i+1],outline2[j+1])$;	
				$i = i+2$;
			}{
				$outline[k] = outline2[j]$;
				$k = k + 1$;
				$outline[k] = max(outline1[i+1],outline2[j+1])$;
				$j = j+2$;
			}
		}
		\While{$i<size(outline1)$}{将outline1的剩余部分赋值给outline;}
		\While{$j<size(outline2)$}{将outline2的剩余部分赋值给outline;}
		return outline;
	\end{algorithm}
\end{solution}


\end{enumerate}

%========================================================================
\end{document}
